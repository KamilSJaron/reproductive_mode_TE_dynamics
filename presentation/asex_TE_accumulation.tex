\pdfminorversion=4

\documentclass[xcolor=dvipsnames]{beamer}
\mode<presentation>
{
\setbeamercolor{structure}{fg=Yellow!50!black}
\usetheme{Madrid}
\setbeamercovered{transparent}
\usefonttheme{professionalfonts}
}


\usepackage{ucs}
\usepackage[utf8]{inputenc}
\usepackage[english]{babel}
\usepackage{palatino}
\usepackage{graphicx}
\usepackage{multicol}
\usepackage{verbatim}
\usepackage{multirow}
\usepackage{url}
\usepackage{txfonts} % medbullt
\usepackage{marvosym} % sex asex symbols
\usepackage{amsmath,amsfonts,bm} % math fonts, amsfonts,amsthm,
\usepackage{array}
\urlstyle{sf}

%%%%% SET TO CORRECT PATHS %%%
% popular choices :
%	{/Volumes/dump/pictures/pool/ampiho/}
%	{/Volumes/dump/pictures/pool/sex/}{/Volumes/dump/pictures/pool/stick/}
\graphicspath{
	{/Volumes/dump/pictures/pool/}
	{../figures/}
}



\newcolumntype{L}[1]{>{\raggedright\let\newline\\\arraybackslash\hspace{0pt}}m{#1}}
\newcolumntype{C}[1]{>{\centering\let\newline\\\arraybackslash\hspace{0pt}}m{#1}}
\newcolumntype{R}[1]{>{\raggedleft\let\newline\\\arraybackslash\hspace{0pt}}m{#1}}

\newcommand{\btVFill}{\vskip0pt plus 1filll}

\newcommand{\textSlide}[1]{
	\begin{frame}
		\Huge
		\begin{center}
			#1
		\end{center}
	\end{frame}
}

\newcommand{\textCapSlide}[2]{
    \begin{frame}
        \frametitle{#1}
        \Huge
        \begin{center}
            #2
        \end{center}
    \end{frame}
}


\newcommand{\picSlideHorizontal}[1]{
	\textSlide{
        \includegraphics[width=0.9\textwidth]{#1}
    }
}

\newcommand{\picSlideVertical}[1]{
    \textSlide{
		\includegraphics[height=0.9\textheight]{#1}
    }
}

\newcommand{\FourTilesSlide}[4]{
	\begin{frame}
		\begin{center}
			\begin{tabular}{ C{5cm} C{5cm} }
				#1 & #2 \\
				#3 & #4 \\
			\end{tabular}
		\end{center}
	\end{frame}
}

\newcommand{\simulationSlide}[3]{
	\begin{frame}
		\frametitle{#1}
		\begin{center}
			\begin{tabular}{ C{7cm} C{3cm} }
				\includegraphics[height=0.8\textheight]{#2} &
				\begin{itemize}
					#3
				\end{itemize} \\
			\end{tabular}
		\end{center}
	\end{frame}
}

\newcommand{\picCapSlideHorizontal}[2]{
	\begin{frame}
		\frametitle{#1}
		\begin{center}
			\includegraphics[width=0.8\textwidth]{#2}
		\end{center}
	\end{frame}
}

\newcommand{\picCapSlideVertical}[2]{
	\begin{frame}
		\frametitle{#1}
		\begin{center}
			\includegraphics[height=0.8\textheight]{#2}
		\end{center}
	\end{frame}
}

\newcommand{\turnFooterOn}{
	\setbeamertemplate{footline}{
		\raisebox{5pt}{
			\makebox[\paperwidth]{
			\tiny
			\begin{tabular}{c c}
				\textcolor{sex_red}{$\medbullet$}  & \Female~\Male \\
				\textcolor{asex_blue}{$\medbullet$}  & \Female \\
			\end{tabular}
			\hfill
			\makebox[13pt]{\scriptsize\insertframenumber}
			}
		}
	}
}

\newcommand{\turnFooterOff}{
	\setbeamertemplate{footline}{\raisebox{5pt}{\makebox[\paperwidth]{\hfill\makebox[13pt]{\scriptsize\insertframenumber}}}}
}


\definecolor{asex_blue}{RGB}{146,197,222}
\definecolor{sex_red}{RGB}{214,96,77}
% sex_red <- "#D6604DED"
% asex_blue <- "#92C5DECD"

%%%%%% SET TITLE
\title{ Accumulation of divergence between haplotypes in the absence of sex }
\author{ Kamil S. Jaron }

% destroy that thing at the bottom
\usenavigationsymbolstemplate{}
\defbeamertemplate*{footline}{mytheme}{\raisebox{5pt}{\makebox[\paperwidth]{\hfill\makebox[13pt]{\scriptsize\insertframenumber}}}}

% \setbeamertemplate{footline}[frame number]


\begin{document}

\turnFooterOff
\Large

\textSlide{ Modeling of TE loads in sexual and asexual yeast. }

\picSlideVertical{yeast_setting}

\picSlideVertical{yiest_TE}

\picCapSlideVertical{ Estimated from data }{TE_inference_over_time}

\textSlide{ Model \large \\
\begin{itemize}
    \item fitness proportional to number of TEs
    \item fixed population size, fitness is relative
    \item transposition happens in haploid state
    \item recombination is a poisson process
	\item chiasmata are uniformly distributed on chromosome
\end{itemize} }

\picCapSlideVertical{TE load $\rightarrow$ fitness}{default_fitness_function}

\picCapSlideVertical{fitness $\rightarrow$ P(being parent)}{relative_fitness_expl}

\turnFooterOn

\simulationSlide{ Naive-brute asex :-) }{004_basic_asex}{
\large
\item $t_r = 10e-6$
\item $e_r = 10e-6$
}

\simulationSlide{ The same but sex }{003_basic_sex}{
\large
\item $t_r = 10e-6$
\item $e_r = 10e-6$
}

\turnFooterOff

\textSlide{ How asex sim will look like for parameters of equillibrium sim? }

\turnFooterOn

\simulationSlide{ Sexual equillibrium }{013_sex_equil}{
\large
\item $t_r = 4e-3$
\item $e_r = 10e-6$
}

\simulationSlide{ But asexual still accumulate }{015_seq_equil_asex_sim}{
\large
\item $t_r = 4e-3$
\item $e_r = 10e-6$
}

\textSlide{ What part of reality the model is missing? \large \\
\begin{itemize}
    \item Too strong selection?
    \item Is transition rate of TE adaptive?
	\item How to model mutations in transition rate?
    \item Given the inference of TEs form yeast data. Should we measure just proportion of initial TEs?
\end{itemize} }

\textSlide{\huge Thank you for your attention! \vspace{1.5cm} \\ \includegraphics[width=0.6\textwidth]{logos/wiki_logo_v2.png}}

\picSlideVertical{fitness_parametrization}

\end{document}
